\section{基于能量收割的半主动控制磁流变阻尼器}
\subsection{理论模型}
如前所述,结构振动控制作为结构抗震、抗风的核心技术,具有极强的工程需求与研究价值。半主动控制作为结构振动控制中的一类方式,兼具能耗小与智能切换控制状态的优点,是目前最具有工程应用前景的一种结构控制方法。磁流变阻尼器作为一种相对新型的智能阻尼器,具有构造简单,控制能耗低、响应速度快、出力大及可连续调节阻尼等优点,目前也正越来越多地应用于工程领域。结合半主动控制与磁流变(MR)阻尼器无疑为优化结构振动控制提供了一个高效可行的方向。然而,值得注意的是,尽管半主动控制具有能耗小的优点,但其也需要能量输入,同时磁流变阻尼器则是一种典型的直接耗能减振装置,同样也需要一定的能量输入,当其应用于桥梁等户外结构或在地震条件下工作时,不方便或可靠性不能保证的供电,大大制约了半主动控制磁流变阻尼器的应用。事实上,结构减振的过程,就是将结构振动能量耗散的过程,假如将振动的能量收集起来应用于半主动控制磁流变阻尼器的能源供给,上述制约也将迎刃而解。

因此,我们小组设计了一个基于能量收割的半主动控制磁流变阻尼器系统,利用能量收割的原理将振动能量转化为电能供给半主动控制系统与磁流变阻尼器,充分发挥半主动控制与磁流变阻尼器的优势,以达到更优的结构减振效果。系统理论模型如图XXX所示。

\begin{figure}[H]
\caption{系统理论模型概念图}
\end{figure}

系统工作原理如下:XXXX

由此可见,该系统主要分为MR阻尼器、能量收割模块与半主动控制模块三个部分。由于MR阻尼器的现有研究与工程应用已经相对成熟,我们并未在MR阻尼器的构造上进行过多地创新,仅根据结构特点与需求设计了“双出型”磁流变阻尼器,对其力学性质进行数值模拟与试验测试。针对能量收割模块,我们分析讨论在结构振动控制领域能量收割的一些方式,为验证能量收割的可行性,测算能量转换效率,制作了一个直线型电磁感应能量收割装置进行试验。对于半主动控制模块,XXXX

\subsection{振动能量收割原理及实现方式}

所谓能量收割(Energy Harvesting),是指通过特定方式将通常直接消耗或废弃的能量转化为其他更有用的能量形式。而土木工程结构由于风荷载、地震激励或人致激励产生的振动能就是一种通常直接消耗或废弃的能量,由于结构的振动无所不在,从理论上而言,这种能量非常大,这也意味着土木工程结构能够收割的能量非常大。然而,相对于航天航空、机械、电子等领域的振动能量,土木工程结构的振动能量属于低频能量,往往只有几赫兹,这使得广泛应用于高频能量收割的压电材料并不能够很好地适用于土木工程领域。

目前在土木工程领域已有的能量收割研究主要有两个方向:将振动能量转化为将振动能量转化为液压能与将振动能量转化为电能。由于半主动控制磁流变阻尼器要求的能量输入形式为电能,我们选择的能量收割途径为将振动的机械能转化为电能。静电、压电方式与电磁感应是将振动机械能转化为电能的三种形式,然而由于静电式转换机械能时需要外部电源提供静电场,与我们的目的相悖,而压电式能量转换系统并不能够很好地适用于低频的振动机械能转换。因此,我们选用电磁感应能量转换系统来实现能量收割。

电磁感应能量转换系统,通俗而言,就是应用电磁感应技术进行发电的发电机。电磁感应技术是一项非常成熟,应用发展也非常完善的发电技术。相比于压电能量捕获技术,电磁感应技术电能转换效率更高、发电量更大、理论技术也更为完善,最重要的是,它能够适应土木工程结构低频振动的特点,相对高效地进行结构振动能量收割。目前商用的发电机多为旋转式发电机,但考虑到结构振动是线性运动,如使用旋转式发电机,还需要增加转换装置,制作复杂,也使得机械能耗散加大,降低能量转换效率,所以我们决定根据结构特点,设计直线型发电机用于能量收割。

\subsection{半主动控制算法及系统}
控制算法选择的好坏直接影响控制系统的性能。对磁流变阻尼结构来货,由于磁流变阻尼器是通过调整磁场的强度来调整产生的阻尼里,接着通过调节事假在磁流变阻尼器的电压(电流)大小使其产生的阻尼里趋向于最优的控制力。有效改变电压(电流)的大小,是利用磁流变阻尼器实现有效控制的关键所在。国内外学者根据磁流变阻尼器的特显,提出了很多种控制策略。

\subsubsection{恒定加压式}
这种控制算法主要包括了 Passive-off, Passive-on 两种形式,分别指对磁流变阻尼器不施加电压和施加最大电压。

在这种恒定加压式控制算法下,磁流变阻尼器相当于一个被动阻尼器,其形式简单,操作便捷,但存在着很大的缺憾:
\begin{enumerate}
\item Passive-off 策略提供的阻尼力较小,不能使磁流变阻尼器发挥出真正的性能。
\item Passive-on 策略提供的阻尼力较大,阻尼器的内力很少有机会超过磁流变阻尼器的屈服力,因此磁流变阻尼器仅起到一个增大结构刚度的作用,未祈祷耗能减振的作用,在地震时因为刚度的增大反而可能加大地震响应。
\end{enumerate}

\subsubsection{离散加压式}

\subsection{实验模型设计}